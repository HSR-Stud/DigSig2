\section{FIR Digital Filter Design\buch{Chapter 10}}

\subsection{Windowing Methods}
\subsubsection{Ideal Filters}
Ideal frequency responses (Fig. \ref{fig:freqresp}) are transformed into infinite impulse responses $d(k)$ (eq: \ref{eq:impresp}), which are then made finite using a particular window.

\begin{figure}[htp]
\begin{subfigure}{0.49\textwidth}
\centering
\newcommand{\coordinates}{coordinates {
(-3.14,1) (-1.569,1) (-1.571,0)
( 1.571,0) (1.569,1) ( 3.14,1)
};}
\begin{tikzpicture}[
trim axis left,
trim axis right
]
\begin{axis}[
	every axis plot post/.append style={mark=none},
	width=0.5\textwidth, height=0.25\textwidth,
	scale only axis,
	xmin=-3.14, xmax=3.14,
	ymin=0, ymax=1.2,
	xlabel=$\omega$,
	ylabel=$D(\omega)$,
	axis x line=bottom, axis y line=middle, enlargelimits=upper,
	xtick={-3.14,-1.57,0,1.57, 3.14}, xticklabels={$-\pi$, $-\omega_c$, 0, $\omega_c$, $\pi$}
	]
	\expandafter\addplot\coordinates
\end{axis}
\end{tikzpicture}
\caption{Highpass}
\end{subfigure}
\begin{subfigure}{0.49\textwidth}
\centering
\newcommand{\coordinates}{coordinates {(-3.14,0) (-1.569,0) (-1.571,1) (1.571,1) (1.569,0) (3.14,0)};}
\begin{tikzpicture}[
trim axis left,
trim axis right
]
\begin{axis}[
	every axis plot post/.append style={mark=none},
	width=0.5\textwidth, height=0.25\textwidth,
	scale only axis,
	xmin=-3.14, xmax=3.14,
	ymin=0, ymax=1.2,
	xlabel=$\omega$,
	ylabel=$D(\omega)$,
	axis x line=bottom, axis y line=middle, enlargelimits=upper,
	xtick={-3.14,-1.57,0,1.57, 3.14}, xticklabels={$-\pi$, $-\omega_c$, 0, $\omega_c$, $\pi$}
	]
	\expandafter\addplot\coordinates
\end{axis}
\end{tikzpicture}
\caption{Lowpass}
\end{subfigure}

\begin{subfigure}{0.49\textwidth}
\centering
\newcommand{\coordinates}{coordinates {
(-3.14,1) (-2.001,1) (-1.999,0) (-1.001,0) (-0.999,1)
(0.999,1) ( 1.001,0) ( 1.999,0) ( 2.001,1) ( 3.14,1)};}
\begin{tikzpicture}[
trim axis left,
trim axis right
]
\begin{axis}[
	every axis plot post/.append style={mark=none},
	width=0.5\textwidth, height=0.25\textwidth,
	scale only axis,
	xmin=-3.14, xmax=3.14,
	ymin=0, ymax=1.2,
	xlabel=$\omega$,
	ylabel=$D(\omega)$,
	axis x line=bottom, axis y line=middle, enlargelimits=upper,
	xtick={-3.14,-2, -1,0,1,2, 3.14}, xticklabels={$-\pi$, $-\omega_b$, $-\omega_a$, 0, $\omega_a$, $\omega_b$, $\pi$}
	]
	\expandafter\addplot\coordinates
\end{axis}
\end{tikzpicture}
\caption{Bandpass}
\end{subfigure}
\begin{subfigure}{0.49\textwidth}
\centering
\newcommand{\coordinates}{coordinates {
(-3.14,0) (-2.001,0) (-1.999,1) (-1.001,1) (-0.999,0)
(0.999,0) (1.001,1) (1.999,1) (2.001,0) ( 3.14,0)};}
\begin{tikzpicture}[
trim axis left,
trim axis right
]
\begin{axis}[
	every axis plot post/.append style={mark=none},
	width=0.5\textwidth, height=0.25\textwidth,
	scale only axis,
	xmin=-3.14, xmax=3.14,
	ymin=0, ymax=1.2,
	xlabel=$\omega$,
	ylabel=$D(\omega)$,
	axis x line=bottom, axis y line=middle, enlargelimits=upper,
	xtick={-3.14,-2, -1,0,1,2, 3.14}, xticklabels={$-\pi$, $-\omega_b$, $-\omega_a$, 0, $\omega_a$, $\omega_b$, $\pi$}
	]
	\expandafter\addplot\coordinates
\end{axis}
\end{tikzpicture}
\caption{Bandstop}
\end{subfigure}

\begin{subfigure}{0.49\textwidth}
\centering
\newcommand{\coordinates}{coordinates {(-3.14,-1) (3.14,1)};}
\begin{tikzpicture}[
trim axis left,
trim axis right
]
\begin{axis}[
	every axis plot post/.append style={mark=none},
	width=0.5\textwidth, height=0.25\textwidth,
	scale only axis,
	xmin=-3.14, xmax=3.14,
	xlabel=$\omega$,
	ylabel=$D(\omega)/j$,
	axis x line=middle, axis y line=middle, enlargelimits=upper,
	xtick={-3.14,0,3.14}, xticklabels={$-\pi$,  0, $\pi$}
	]
	\expandafter\addplot\coordinates
\end{axis}
\end{tikzpicture}
\caption{Differentiator}
\end{subfigure}
\begin{subfigure}{0.49\textwidth}
\centering
\newcommand{\coordinates}{coordinates {(-3.14,1) (-0.001,1) (0.001,-1) (3.14,-1) };}
\begin{tikzpicture}[
trim axis left,
trim axis right
]
\begin{axis}[
	every axis plot post/.append style={mark=none},
	width=0.5\textwidth, height=0.25\textwidth,
	scale only axis,
	xmin=-3.14, xmax=3.14,
	xlabel=$\omega$,
	ylabel=$D(\omega)/j$,
	axis x line=middle, axis y line=middle, enlargelimits=upper,
	xtick={-3.14,0,3.14}, xticklabels={$-\pi$,  0, $\pi$}
	]
	\expandafter\addplot\coordinates
\end{axis}
\end{tikzpicture}
\caption{Hilbert transformer}
\end{subfigure}

\caption{Ideal frequency responses\buchSeite{532}}
\label{fig:freqresp}
\end{figure}


\begin{equation}
d(k) = \int_{-\pi}^{\pi}D(\omega) e^{j\omega k}\frac{d\omega}{2\pi} \label{eq:impresp}
\end{equation}

\begin{align*}
&\text{lowpass filter} && d(k) = \frac{sin(\omega_ck)}{\pi k} \\
&\text{highpass filter} && d(k) = \delta(k) - \frac{sin(\omega_ck)}{\pi k} \\
&\text{bandpass filter} && d(k) = \frac{sin(\omega_bk) -sin(\omega_ak)}{\pi k} \\
&\text{bandstop filter} && d(k) = \delta(k) - \frac{sin(\omega_bk) -sin(\omega_ak)}{\pi k} \\
&\text{differentiator} && d(k) = \frac{cos(\pi k)}{k}-\frac{sin(\pi k)}{\pi k^2}\\
&\text{Hilber transformer} && d(k) = \frac{1-cos(\pi k)}{\pi k}
\end{align*}

\subsubsection{Window comparison}
\begin{tabular}{|l|c|c|c|c|c|c|}
	\hline
	Window & $\delta$ & $A_{stop}$ & $A_{pass}$ & $D$ & $R$ & $c$ \\
	\hline
	Rectangular & 8.9\% & $-21$ dB & $1.55$ dB & 0.92 & $-13$ dB & 1\\
	Hamming & 0.2\% & $-54$ dB & $0.03$ dB & 3.21 & $-40$ dB & 2\\
	Kaiser & variable $\delta$ & $-20log_{10}\delta$ & $17.372\delta$ & $D = \left\{
						\begin{array}{l l}
							\frac{A-7.95}{14.36}& \quad \text{if } A>21\\
							0.922 & \quad \text{if } A \leq 21
						\end{array} \right.$ & variable R & $6(R+12)/55$\\
	\hline
\end{tabular}

\subsubsection{Rectangular Window\buchSeite{535}}
\begin{itemize}
	\item Simplest window
	\item truncate the infinite impulse response: $-M < k < M$
	\item $\rightarrow d = [d_{-M}, ..., d_{-1}, d_0, d_1, ..., d_M]$
	\item The resulting FIR Filter must be delayed by M samples \newline
	$\rightarrow h = d = [h_0, h_{M-1}, h_M, h_{M+1}, ..., h_{2M}]$
\end{itemize}


\subsubsection{Hamming Window\buchSeite{540}}
\begin{itemize}
	\item Improved passband and stopband ripple suppression at the cost of a wider transition width.
	\item Hamming window impulse response:
	\begin{align*}
		w(n) = 0.54 -0.46cos\left(\frac{2\pi n}{N-1}\right),&&n= 0,1,\dots,N-1
	\end{align*}
	\item Filter impuls response:
	\begin{align*}
		h(n) = w(n)d(n-M),&&-\infty < n < \infty
	\end{align*}
	\item e.g. lowpass filter
	\begin{align*}
		h(n) = \left[0.54 -0.46cos\left(\frac{2\pi n}{N-1}\right)\right]\cdot\frac{sin(\omega_c(n-M))}{\pi (n-M)}
	\end{align*}
\end{itemize}


\subsubsection{Kaiser Window\buchSeite{541}}
\begin{tabular}{|l|l|}
	\hline
	cutoff frequency & $f_c=\frac{1}{2}(f_{pass}+f_{stop})$ \\ \hline
	transition width & $\Delta f = f_{stop}-f_{pass}$\\ \hline
	passband/stopband frequency & $f_{pass} = f_c - \frac{1}{2}\Delta f,\quad f_{stop}= fc+\frac{1}{2}\Delta f$\\ \hline
	digital frequencies & $\omega_{pass}=\frac{2\pi f_{pass}}{f_s},\quad \omega_{stop}=\frac{2\pi f_{stop}}{f_s},\quad
						  \omega_{c}=\frac{2\pi f_c}{f_s},\quad \Delta\omega=\frac{2\pi \Delta f}{f_s}$\\ \hline
	passband/stopband overshoot & $A_{pass}=20log_{10}\left(\frac{1+\delta_{pass}}{1-\delta_{pass}}\right),\quad A_{stop}=-20log_{10}\delta_{stop}$\\
								& $\delta_{pass}=\frac{10^{A_{pass}/20}-1}{10^{A_{pass}/20}+1},\quad \delta_{stop} = 10^{-A_{stop}/20}$ \\ \hline
	window parameters &
		$\alpha = \left\{
					\begin{array}{l l}
						0.1102(A-8.7)& \quad \text{if } A \geq 50\\
						0.5842(A-21)^{0.4}+0.07886(A-21) & \quad \text{if }21 < A < 50\\
						0 & \quad \text{if } A \leq 21
					\end{array} \right. $\\
	&	$D = \left\{
					\begin{array}{l l}
						\frac{A-7.95}{14.36}& \quad \text{if } A>21\\
						0.922 & \quad \text{if } A \leq 21
					\end{array} \right. $\\ \hline
	window function & $w(n)=\frac{I_0(\alpha\sqrt{1-(n-M)^2/M2})}{I_0(\alpha)}=\frac{I_0(\alpha\sqrt{n(2M-n)}/M)}{I_0(\alpha)},$ \\
	& $I_0=\text{modified Bessel function of the first kind and 0th order.}$\\
	\hline						  
\end{tabular}
\subsubsection{Design steps for a lowpass filter with Kaiser Window\buchSeite{545}}
\begin{enumerate}
\item Calculate $f_c$, $\Delta f$ and $\omega_c$.
\item Calculate $\delta_{pass}$ and $\delta_{stop}$.
\item Calculate $\delta=min(\delta_{pass},\delta_{stop})$ and $A=-20log_{10}\delta$ in dB. \newline
	  The filter will have equal passband and stopband ripples. Therefore the design must be based on the smaller of the two ripples.
\item Calculate $\alpha$ and $D$.
\item Calculate the filter length $N=\frac{Df_s}{\Delta f}+1$ and round it up to the next \textbf{odd} number. Calculate $M=(N-1)/2$.
\item Calculate the window function $w(n),\quad n=0,1,\dots,N-1$
\item Calculate the windowed impuls response $h(n)=w(n)d(n-M)$.
\end{enumerate}

\subsubsection{Rules for different filter types}
\paragraph{Highpass filter}~\\
The role of $f_{pass}$ and $f_{stop}$ are interchanged, therefore $\Delta f = f_{pass}-f_{stop}$
\paragraph{Bandpass filter}~\\
The final filter design will have equal transition widths, therefore $\Delta f = min(\Delta f_a, \Delta f_b)$ with $\Delta f_a=f_{pa}-f_{sa}$ and $\Delta f_b=f_{pb}-f_{sb}$.

\subsubsection{Kaiser Window for Spectral Analysis\buchSeite{555ff}}

\subsection{Frequency Sampling Method\buchSeite{558}}

\subsection{Other FIR Design Methods}

