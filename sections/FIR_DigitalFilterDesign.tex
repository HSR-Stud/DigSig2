\section{FIR Digital Filter Design\buch{Chapter 10}}

\subsection{Windowing Methods}
\subsubsection{Ideal Filters}
Ideal frequency responses (Fig. \ref{fig:freqresp}) are transformed into infinite impulse responses $d(k)$ (eq: \ref{eq:impresp}), which are then made finite using a particular window.

\begin{figure}[htp]
\begin{subfigure}{0.49\textwidth}
\centering
\newcommand{\coordinates}{coordinates {
(-3.14,1) (-1.569,1) (-1.571,0)
( 1.571,0) (1.569,1) ( 3.14,1)
};}
\begin{tikzpicture}[
trim axis left,
trim axis right
]
\begin{axis}[
	every axis plot post/.append style={mark=none},
	width=0.5\textwidth, height=0.25\textwidth,
	scale only axis,
	xmin=-3.14, xmax=3.14,
	ymin=0, ymax=1.2,
	xlabel=$\omega$,
	ylabel=$D(\omega)$,
	axis x line=bottom, axis y line=middle, enlargelimits=upper,
	xtick={-3.14,-1.57,0,1.57, 3.14}, xticklabels={$-\pi$, $-\omega_c$, 0, $\omega_c$, $\pi$}
	]
	\expandafter\addplot\coordinates
\end{axis}
\end{tikzpicture}
\caption{Highpass}
\end{subfigure}
\begin{subfigure}{0.49\textwidth}
\centering
\newcommand{\coordinates}{coordinates {(-3.14,0) (-1.569,0) (-1.571,1) (1.571,1) (1.569,0) (3.14,0)};}
\begin{tikzpicture}[
trim axis left,
trim axis right
]
\begin{axis}[
	every axis plot post/.append style={mark=none},
	width=0.5\textwidth, height=0.25\textwidth,
	scale only axis,
	xmin=-3.14, xmax=3.14,
	ymin=0, ymax=1.2,
	xlabel=$\omega$,
	ylabel=$D(\omega)$,
	axis x line=bottom, axis y line=middle, enlargelimits=upper,
	xtick={-3.14,-1.57,0,1.57, 3.14}, xticklabels={$-\pi$, $-\omega_c$, 0, $\omega_c$, $\pi$}
	]
	\expandafter\addplot\coordinates
\end{axis}
\end{tikzpicture}
\caption{Lowpass}
\end{subfigure}

\begin{subfigure}{0.49\textwidth}
\centering
\newcommand{\coordinates}{coordinates {
(-3.14,1) (-2.001,1) (-1.999,0) (-1.001,0) (-0.999,1)
(0.999,1) ( 1.001,0) ( 1.999,0) ( 2.001,1) ( 3.14,1)};}
\begin{tikzpicture}[
trim axis left,
trim axis right
]
\begin{axis}[
	every axis plot post/.append style={mark=none},
	width=0.5\textwidth, height=0.25\textwidth,
	scale only axis,
	xmin=-3.14, xmax=3.14,
	ymin=0, ymax=1.2,
	xlabel=$\omega$,
	ylabel=$D(\omega)$,
	axis x line=bottom, axis y line=middle, enlargelimits=upper,
	xtick={-3.14,-2, -1,0,1,2, 3.14}, xticklabels={$-\pi$, $-\omega_b$, $-\omega_a$, 0, $\omega_a$, $\omega_b$, $\pi$}
	]
	\expandafter\addplot\coordinates
\end{axis}
\end{tikzpicture}
\caption{Bandpass}
\end{subfigure}
\begin{subfigure}{0.49\textwidth}
\centering
\newcommand{\coordinates}{coordinates {
(-3.14,0) (-2.001,0) (-1.999,1) (-1.001,1) (-0.999,0)
(0.999,0) (1.001,1) (1.999,1) (2.001,0) ( 3.14,0)};}
\begin{tikzpicture}[
trim axis left,
trim axis right
]
\begin{axis}[
	every axis plot post/.append style={mark=none},
	width=0.5\textwidth, height=0.25\textwidth,
	scale only axis,
	xmin=-3.14, xmax=3.14,
	ymin=0, ymax=1.2,
	xlabel=$\omega$,
	ylabel=$D(\omega)$,
	axis x line=bottom, axis y line=middle, enlargelimits=upper,
	xtick={-3.14,-2, -1,0,1,2, 3.14}, xticklabels={$-\pi$, $-\omega_b$, $-\omega_a$, 0, $\omega_a$, $\omega_b$, $\pi$}
	]
	\expandafter\addplot\coordinates
\end{axis}
\end{tikzpicture}
\caption{Bandstop}
\end{subfigure}

\begin{subfigure}{0.49\textwidth}
\centering
\newcommand{\coordinates}{coordinates {(-3.14,-1) (3.14,1)};}
\begin{tikzpicture}[
trim axis left,
trim axis right
]
\begin{axis}[
	every axis plot post/.append style={mark=none},
	width=0.5\textwidth, height=0.25\textwidth,
	scale only axis,
	xmin=-3.14, xmax=3.14,
	xlabel=$\omega$,
	ylabel=$D(\omega)/j$,
	axis x line=middle, axis y line=middle, enlargelimits=upper,
	xtick={-3.14,0,3.14}, xticklabels={$-\pi$,  0, $\pi$}
	]
	\expandafter\addplot\coordinates
\end{axis}
\end{tikzpicture}
\caption{Differentiator}
\end{subfigure}
\begin{subfigure}{0.49\textwidth}
\centering
\newcommand{\coordinates}{coordinates {(-3.14,1) (-0.001,1) (0.001,-1) (3.14,-1) };}
\begin{tikzpicture}[
trim axis left,
trim axis right
]
\begin{axis}[
	every axis plot post/.append style={mark=none},
	width=0.5\textwidth, height=0.25\textwidth,
	scale only axis,
	xmin=-3.14, xmax=3.14,
	xlabel=$\omega$,
	ylabel=$D(\omega)/j$,
	axis x line=middle, axis y line=middle, enlargelimits=upper,
	xtick={-3.14,0,3.14}, xticklabels={$-\pi$,  0, $\pi$}
	]
	\expandafter\addplot\coordinates
\end{axis}
\end{tikzpicture}
\caption{Hilbert transformer}
\end{subfigure}

\caption{Ideal frequency responses}
\label{fig:freqresp}
\end{figure}


\begin{equation}
d(k) = \int_{-\pi}^{\pi}D(\omega) e^{j\omega k}\frac{d\omega}{2\pi} \label{eq:impresp}
\end{equation}

\subsubsection{Rectangular Window}
\subsubsection{Hamming Window}

\subsection{Kaiser Window}
\begin{align*}
	\alpha = \left\{
		\begin{array}{l l}
			0.1102(A-8.7)& \quad \text{if } A \geq 50\\
			0.5842(A-21)^{0.4}+0.07886(A-21) & \quad \text{if }21 < A < 50\\
			0 & \quad \text{if } A \leq 21
		\end{array} \right.
\end{align*}
\subsubsection{Kaiser Window for Filter Design}


\textbf{Regeln:}\\
pass und stopp Band sind identisch, sind sie unterschiedlich vorgegeben muss die härtere Bedingung verwendet werden \\
\begin{align*}
\delta=min(\delta_{pass},\delta_{stop})
\end{align*}
schlussendlich wird der Filter in beiden Bändern die gleiche Überschreitung haben
\begin{align*}
A=-20log_{¶10}\delta && \delta=10^{-A/20}
\end{align*}
Das Stopp band hat meistens die strengeren Anforderungen
\subsubsection{Kaiser Window for Spectral Analysis}
\subsection{Frequency Sampling Method}

\subsection{Other FIR Design Methods}

